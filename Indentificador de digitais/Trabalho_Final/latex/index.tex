\hypertarget{index_sec_intro}{}\section{Introdução}\label{index_sec_intro}
O projeto tem o objetivo de identificar uma impressão digital em uma base de dados que acompanha o trabalho. Nesta documentação seram detalhadas as informações necessárias para compilar, executar e usar o programa.

Para facilitar o entendimento a explicação do programa serão divididas em duas etapas. A primeira referente ao processamento de uma foto de um dedo e a segunda referente a classificação. ~\newline
 \hypertarget{index_sec_install}{}\section{Compilando, Executando e Usando o Programa}\label{index_sec_install}
\hypertarget{index_ssec1}{}\subsection{Compilando e Executando}\label{index_ssec1}
Para Compilar e executar o código é necessario entrar na pasta build, e executar os seguintes comandos.


\begin{DoxyItemize}
\item cmake ..
\item make
\item ./\+Digital\+\_\+\+Pattern
\end{DoxyItemize}\hypertarget{index_ssec2}{}\subsection{Menu}\label{index_ssec2}
A imagem Abaixo mostra o menu inicial do programa.



A primeira opção possibilita buscar no banco de dados uma digital com o mesmo padrão de uma imagem desejada. Para facilitar foram carregadas algumas opções de imagem para pesquisar no banco. A imagem abaixo mostra essas opções, e o resultado de uma consulta.



A segunda opção do menu executa a etapa de tratamento de uma imagem. Essa imagem tratada, já acompanha o projeto. Todas as etapas do processamento são exibidas na tela em forma de imagens.

A ultima opção do menu tem o objetivo de fechar o programa.

As imagens referentes a base de dados, esta no diretório src/base, e as demais fotos estão no diretório src/img.\hypertarget{index_sec_summary}{}\section{Resumo}\label{index_sec_summary}
\hypertarget{index_ssec3}{}\subsection{Processamento de imagens}\label{index_ssec3}
Nesta etapa é reslizado o tratamento de uma imagem, para que seja possível realizar a indentificação dos padrões da digital. O fluxograma seguinte exemplifica as etapas.



As imagens seguintes ilustram as etapas mostradas no fluxograma.


\begin{DoxyItemize}
\item A primeira imagem mostra a foto do dedo em escala de cinza.
\end{DoxyItemize}




\begin{DoxyItemize}
\item A segunda mostra a imagem da mascara usada seguimentar o dedo. Threshold. ~\newline
 
\item A terceira imagem mostra a equalização da imagem, realizada para destacar as linhas do deto.
\end{DoxyItemize}




\begin{DoxyItemize}
\item A proxima imagem mostra a imagem binaria.
\end{DoxyItemize}



-\/E a última mostra o esqueleto da imagem.

\hypertarget{index_ssec4}{}\subsection{Identificação da Digital}\label{index_ssec4}
Esta etapa mostra a parte da detecção das caracteristicas e comparação com outras impressões digitais. O fluxograma abaixo mostra as etapas do processo.




\begin{DoxyItemize}
\item A imagem abaixo mostra as Minúcias detectadas usando o Harris Corner.
\end{DoxyItemize}




\begin{DoxyItemize}
\item A Proxima imagem mostra o casamento de duas imagens, realizando as comparações das caracteristicas de cada uma.
\end{DoxyItemize}



\begin{DoxyAuthor}{Author}
Douglas Venâncio 

Filipe Pena 

Marco Vinha 
\end{DoxyAuthor}
\begin{DoxyDate}{Date}
Junho de 2018 
\end{DoxyDate}
